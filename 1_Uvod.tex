\chapter{Úvod}
% \section*{}
Systémová integrace je velmi často skloňovaný pojem. Drtivá většina systémů v dnešní době
není bez integrace schopna fungovat. Schopnost integrace s jinými systémy je žádoucí,
protože čím více se umí software integrovat, tím je flexibilnější. Integrace znamená spojení
menších komponent do vetšího celku. Takový celek potom dokáže efektivně pracovat pomocí
definovaných pravidel pro komunikaci mezi subsytémy. Integraci lze vytvořit na různých
systémových vrstvách.

První počátky integrace sahají do roku 1980, kdy vznikla myšlenka tzv. digitální výroby, která
byla aplikována v továrnách. V devadesátých letech společnosti nakupovaly softwarová řešení
jako je SAP, Oracle ERP, Siebel a další. Tato řešení fungovala dobře jako samostatný
software, vytvořily se kolem nich takzvané informační ostrovy. Ve většině případů každé
řešení produkovalo redundantní data. Ve výsledku při vzniklé změně dat musely být ručně
změněny i v ostatních systémech. Takové řešení bylo těžkopádné a musela přijít změna. Tyto
problémy daly vzniknout rozmachu integrace mezi systémy.

Komunikace mezi komponentami byla často platformově závislá a používaly se pro ni
proprietární protokoly. Postupem času, jak šly informační technologie dopředu a sílil tlak na
vznik technologií umožňujících platformovou nezávislost s otevřenými protokoly pro
komunikaci, jsme ve stavu, kdy máme technologie a standardizované protokoly. S jejich
pomocí vznikají unifikovaná rozhraní systémů, která zjednodušují realizování integrace mezi
softwarovými komponentami.

Protokoly definují jak probíhá komunikace, technologie definují jak vytvářet rozhraní a
klienty k nim, opomíjí se však tvorba, obsah a umístění dokumentace. Přestože poskytovatelé
rozhraní používají jednotné protokoly a technologie, dokumentaci má každý trochu jinou.
Momentálně se můžeme setkat nejčastěji s dokumentací v HTML, dokumentech
kancelářských balíků či PDF dokumentech. Taková dokumentace je tvořena ručně a má řadu
nevýhod. Při její tvorbě může být zanesena chyba, dokumentace nemusí být aktuální a nemusí
odpovídat popisovanému rozhraní. Dále usí existovat člověk, který bude za dokumentaci
zodpovědný. Provádět integraci se systémem mající neodpovídající dokumentaci může
prodloužit dobu integrace a tím vzrostou náklady.

Cílem diplomové práce je navrhnout, implementovat a otestovat nástroj pro automatické
generování dokumentace vzdálených rozhraní. Nástroj by měl odstranit problémy s tvorbou
dokumentace, protože proces její tvorby nebude zahrnovat lidský element a bude zajištěno,
aby se dokumentace aktualizovala vždy se změnou rozhraní. 

\end{document}